\documentclass[uplatex,
  tate,
  book,
  onecolumn,
  paper=a6,
  oneside,
  openany,
  fontsize=9pt,
  jafontsize=9pt,
  % linegap=6pt,
  number_of_lines=14,
  line_length=35zh,
  baselineskip=16pt,
  hanging_punctuation,
]{jlreq}

\usepackage{pxrubrica}
% \usepackage[jis2004,uplatex,deluxe]{otf} %
\usepackage[noalphabet, unicode]{pxchfon}
\setminchofont[0]{GenEiKoburiMin6-R.ttf}

\ModifyPageStyle{plain}{%
  running_head_position=top-center,
  nombre_position=bottom-center,
  odd_running_head={_chapter},
  even_running_head={},
  mark_format={_chapter={#1}}
}
\pagestyle{plain}
\ModifyHeading{chapter}{format={\large\sffamily #2},indent=2zw,lines=4}
\ModifyHeading{section}{format={\normalsize #2},indent=4zw,lines=2}

% pdftocairo -png -gray document.pdf

\begin{document}

\chapter{あなたと2人、夜の闇に浮かぶ}
\noindent
\hskip.1em
「どこに行くの?」

神琳は答えず、ただ黙って前を向いていた。
私の言葉に答えないなんて珍しいなとは思ったけれど、仕方なくそのままその横顔を見つめていた。

夜の電車には私達以外に誰もいなかった。
百合ヶ丘の最寄駅で乗車した時も、人の住む地域に近づいた時も、そしてまた人の少ない地域になった時にも、乗り合わせたのは全部で十人にも満たなかったと思う。

制服の上に地味なコートを着ていたからそれほど目立たなかったはずだ。
向かいに座る人の目線を引いた感触はなかったが、見てみぬふりをしていただけかもしれない。
こんな時間に女の子が荷物も持たず電車に乗っているのは珍しい光景のはずだが、リリィであればそいういうこともあるだろうと思ったのかもしれない。

これから風呂に入るという時間に神琳に声をかけられ、言われるままにここまでついてきたが、未だに説明はない。
私もさっき一度尋ねただけだからもう少し問い質してもよかったが、結局そうしなかった。
神琳が黙っているなら何か理由があるのだろう。
あるいは、ただそんな気になれないだけだろうか。
どちらにしても私も深く聞かなかった。
\newline
「何を見てるの?」
\newline
「外の風景を」

神琳と同じ方向を私も見てみたが、窓には何も映っていなかった。
畑か山に囲まれた田舎道を走っているはずだが、街灯もなく真っ暗なので、窓に映るのは蛍光灯に照らされた車内の光景だけだった。
吊り革と手すりと、がらんとした座席とそこに座る私達。
窓に映る神琳を見つめても目は合わなかったから、もしかしたら神琳は本当に何かを見ていたのかもしれない。
\newline

ガーデンから脱走するリリィの話は時折耳にする。
計画的に、あるいは無計画に。
どちらにしても\tatechuyoko{10}代の少女の浅はかな行動と計画はすぐに看破され、やがて連れ戻された。

ガーデンではそれほど重くはない懲罰が彼女達を待っていた。
人間同士の戦争と違って敵に寝返ることは不可能だからそれは致命的なものではなかったが、規律違反には違いない。

同情の目を向ける人と、蔑んだ目で見る人とは概ね半分ずつだった。
人類の存亡が懸かった崇高な戦いに意気揚々と臨み、そして仲間の死という現実を目にした時の恐怖。
それを知らない者はいなかったが、自分達だけ生き延びようとする者達への蔑みを隠さない者もいた。
いずれにしても構内の掲示板に名前を貼り出された彼女達の名誉は地に落ち、中にはそのまま正規の手順で去る者もいた。
\newline

神琳の手の感触はいつもと同じだった。
震えてもいないし、冷たくもない。

強く握るとそれに応えて握り返してくれる。
互いの指が絡まるのをぼんやりと目で見ていると、自分が感じている温度がどちらのものかわからなくなる。
やがて2人の体温が溶け合って均一になり、身体の境界がなくなってしまうような錯覚に陥る。
私はあえてそれに身を任せ、神琳の方に身を預けた。

それでも、心には明確に境界線が引かれていた。
神琳が感じているのが恐怖か後悔か、あるいは他のどんなものだったとしても、私の胸に伝わってくることはなかった。
私が感じることができるのは、ただ私自身の寂しさだけだった。
\newline

電車が終点に到着し、ドアが開く。
聞いたことのない駅だったが降りない訳にはいかない。
神琳は立ち上がり、私はそれに従った。

その立ち姿に迷いはなかったが、何か目的があってこんなところまで来たようには思えなかった。
無人の寂れた小さな駅。
周りには商店が数軒あるだけで、灯りは全て消えている。
離れたところには何の変哲もない普通の住宅がぽつぽつと立ち並んでいる。
バスロータリーがあるだけましだったが、車の姿は一台もなかった。

やがて電車のドアが閉じて、ゆっくりと闇に消えて行った。
駅のホームの灯りだけが辺りを照らしている。
神琳が近くのベンチに座ったので、私もその横に座った。
肌寒かったので真横にぴったりと座って身体をくっつけた。
\newline
「思ったよりも星がよく見えませんね。駅の灯りが邪魔ですわ」
\newline
「星を見にきたの?」
\newline
「……さぁ、どうでしょう。私にもよくわかりません」
\newline
「これから、どうする?」
\newline
「どうしましょうか。まずは泊まれる場所を探さないといけませんね」
\newline
「ホテルか旅館かあるかな?」
\newline
「なければどこかのお宅にお願いしましょう」
\newline
「泊めてくれるかな?」
\newline
「若い少女を2人、夜中に外に放り出す人間はいないでしょう。できれば猫がいるお宅がいいですね」
\newline
「そうだね。でも、寝てるのを起こしたら悪いかも」

そう言って私達は目を合わせ、そして2人でくすくすと笑った。
話しているうちに握った手が少しだけ暖まった。
\newline

それからまた無言になった。
色々と話題を探したが、話が弾みそうなものは見つからなかった。

明日のことを考えようとすると、結局は元の生活のことに行き当たった。
ヒュージのこと、ガーデンのこと、故郷のこと、レギオンのこと、家族のこと。
それらを全て放り出せる程私達は絶望できていなかった。
私達の持てる全てを賭けて戦って、海のように広い世界に一つの小石を投げ入れて、そしてささやかな波紋を起こす。
それらが重なり合ってやがて波を呼び、いつの日かこの戦いに終止符を打つのだと私達は信じていた。
例えそれを自分の目で見ることができなかったとしても、その覚悟はできていた。

再び神琳と目が合う。
その瞳の輝きが全く揺らいでないのが、少し残念だった。
神琳が絶望するなら、私もそれに付き合うつもりだったのに。
神琳が私だけを求めてくれるなら、私も神琳だけを求めることができるのに。
2人が互いの事だけを求めて生きることができるなら、それはとても素敵なことなのに。

あるいは私が先に絶望すれば神琳もそれに応えてくれるのかもしれなかったけれど、それはできなかった。
わたしの勇気は私のものだけではない。
仲間と家族と、そして神琳から貰ったものだから、私のわがままでそれを放り出す訳にもいかない。
\newline
「そろそろ帰りましょうか。皆さんきっと心配していますわ」
\newline
「どうやって?もう電車終わっちゃったよ」
\newline
「タクシーを呼びましょう。高くつくでしょうが、仕方ありませんわ」
\newline
「そういえば、ここどこだろう」
\newline
「私もわかりませんけれど、きっと駅の名前を言えば伝わるでしょう」

そう言って神琳は携帯電話を取り出して電源を入れ、電話をかけた。
駅の名前を聞いても全くピンとこなかったけれど、一応は電波が通じる地域らしい。
\newline

それからしばらく2人で辺りを歩いた。
寝ている人を起こしたくなかったのであまり喋らず、ただ風景や建物を眺めた。
特に面白いものはなくて住宅が並んでいるだけだったけれど、何故だかそれだけで心が暖まった。
明日の命の心配をすることもなく、日々の生活を営んでいる人達。
彼らが心安らかに眠れることを祈って、その建物の小さな灯りをひとつひとつ見て歩いた。

やがてタクシーが到着し、ガーデンに戻った頃にはもう明け方だった。
教導官からは叱責され、仲間からは怒られて心配され、謹慎の間に反省文を書いて(2人とも初めてだったので苦労した)この一件は終わった。
その日のうちに私達から連絡したのが良かったのか、公に処罰されることもなかった。
家族へ連絡は行かなかったが、妙なルートで伝わるのは避けたかったので自分から連絡した。

そうして私達は元の生活に戻った。

――過酷で儚くて輝きに満ちた、世界を救う生活に。

\end{document}
