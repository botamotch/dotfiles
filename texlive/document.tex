\RequirePackage{plautopatch}
\documentclass[uplatex,
  tate, % 'yoko' で通常の横書き文書
  book,
  onecolumn, % 'twocolumn' で2段組
  paper=a6,  % 紙面サイズを指定
  twoside,
  openany,
  fontsize=9pt,       % 文字サイズ
  jafontsize=9pt,     % 和文字サイズ
  number_of_lines=14, % 行数
  line_length=35zh,   % 一行の文字数
  baselineskip=16pt,  % 行間
  hanging_punctuation,
]{jlreq}
%%% ルビ生成のためのパッケージ
\usepackage{pxrubrica}
%%% 背景色指定のパッケージ
\usepackage{color}
% \definecolor{background}{cmyk}{0.01, 0.01, 0.07, 0.02} % 黄色っぽいの
\definecolor{background}{cmyk}{0.04, 0.01, 0.01, 0.01} % 青っぽいの
% \definecolor{background}{cmyk}{0, 0, 0, 0} % 白
%%% ページの書式を指定
% 例1)偶数/奇数の区別なし。Twitterに上げる時など
\ModifyPageStyle{plain}{%
  running_head_position=top-center,
  nombre_position=bottom-center,
  odd_running_head={_chapter},
  even_running_head={_chapter},
  mark_format={_chapter={#1}}
}
% 例2)偶数/奇数で別の書式にする。見開きの本にする時など
%\ModifyPageStyle{plain}{%
%	running_head_position=top-left,
%	nombre_position=bottom-left,
%	odd_running_head={_chapter},
%	even_running_head={},
%	mark_format={_chapter={#1}}
%}
\pagestyle{plain}
%%% 章・節の書式を指定
\ModifyHeading{chapter}{format={\large\sffamily #2},indent=2zw,lines=4}
\ModifyHeading{section}{format={\normalsize #2},indent=4zw,lines=2}

%%% ここから本文
\begin{document}
\pagecolor{background}

\chapter{先生と私}
\section{一}

\ruby[<>]{私}{わたくし}はその人を常に先生と呼んでいた。 % \ruby[<>] ルビを前後に進入
だからここでもただ先生と書くだけで本名は打ち明けない。
これは世間を\ruby{憚}{はば}かる遠慮というよりも、その方が私にとって自然だからである。
私はその人の記憶を呼び起すごとに、すぐ「先生」といいたくなる。
筆を\ruby{執}{と}っても心持は同じ事である。
よそよそしい\ruby[g]{頭文字}{かしらもじ} などはとても使う気にならない。 % \ruby[g] 漢字が複数の場合はグループ化する

私が先生と知り合いになったのは\ruby[g]{鎌倉}{かまくら}である。
その時私はまだ若々しい書生であった。
暑中休暇を利用して海水浴に行った友達からぜひ来いという\ruby[g]{端書}{はがき}を受け取ったので、私は多少の金を\ruby[g]{工面}{くめん}して、出掛ける事にした。
私は金の工面に\ruby{二}{に}、\ruby[g]{三日}{さんち}を費やした。
ところが私が鎌倉に着いて三日と\ruby{経}{た}たないうちに、私を呼び寄せた友達は、急に国元から帰れという電報を受け取った。
電報には母が病気だからと断ってあったけれども友達はそれを信じなかった。
友達はかねてから国元にいる親たちに\ruby{勧}{すす}まない結婚を\ruby{強}{し}いられていた。
彼は現代の習慣からいうと結婚するにはあまり年が若過ぎた。
それに\ruby[g]{肝心}{かんじん}の当人が気に入らなかった。
それで夏休みに当然帰るべきところを、わざと避けて東京の近くで遊んでいたのである。
彼は電報を私に見せてどうしようと相談をした。
私にはどうしていいか分らなかった。
けれども実際彼の母が病気であるとすれば彼は固より帰るべきはずであった。
それで彼はとうとう帰る事になった。
せっかく来た私は一人取り残された。

学校の授業が始まるにはまだ\ruby[g]{大分}{だいぶ}\ruby[g]{日数}{ひかず}があるので鎌倉におってもよし、帰ってもよいという境遇にいた私は、当分元の宿に\ruby{留}{と}まる覚悟をした。
友達は中国のある資産家の\ruby[g]{息子}{むすこ}で金に不自由のない男であったけれども、学校が学校なのと年が年なので、生活の程度は私とそう変りもしなかった。
したがって\ruby[g]{一人}{ひとり}ぼっちになった私は別に\ruby[g]{恰好}{かっこう}な宿を探す面倒ももたなかったのである。

宿は鎌倉でも\ruby[g]{辺鄙}{へんぴ}な方角にあった。
\ruby[g]{玉突}{たまつ}きだのアイスクリームだのというハイカラなものには長い\ruby{畷}{なわて}を一つ越さなければ手が届かなかった。
車で行っても二十銭は取られた。
けれども個人の別荘はそこここにいくつでも建てられていた。
それに海へはごく近いので海水浴をやるには至極便利な地位を占めていた。

私は毎日海へはいりに出掛けた。
古い\ruby{燻}{くす}ぶり返った\ruby[g]{藁葺}{わらぶき}の\ruby{間}{あいだ}を通り抜けて\ruby{磯}{いそ}へ下りると、この\ruby{辺}{へん}にこれほどの都会人種が住んでいるかと思うほど、避暑に来た男や女で砂の上が動いていた。
ある時は海の中が\ruby[g]{銭湯}{せんとう}のように黒い頭でごちゃごちゃしている事もあった。
その中に知った人を一人ももたない私も、こういう賑やかな景色の中に裹まれて、砂の上に寝そべってみたり、\ruby[g]{膝頭}{ひざがしら}を波に打たしてそこいらを\ruby{跳}{は}ね\ruby{廻}{まわ}るのは愉快であった。

私は実に先生をこの\ruby[g]{雑沓}{ざっとう}の\ruby{間}{あいだ}に見付け出したのである。
その時海岸には\ruby[g]{掛茶屋}{かけぢゃや}が二軒あった。
私はふとした\ruby[g]{機会}{はずみ}からその一軒の方に行き慣れていた。
\ruby[g]{長谷辺}{はせへん}に大きな別荘を構えている人と違って、\ruby[g]{各自}{めいめい}に専有の\ruby[g]{着換場}{きがえば}を\ruby[<>]{拵}{こしら}えていないここいらの避暑客には、ぜひともこうした共同着換所といった\ruby{風}{ふう}なものが必要なのであった。
彼らはここで茶を飲み、ここで休息する\ruby{外}{ほか}に、ここで海水着を洗濯させたり、ここで\ruby{鹹}{しお}はゆい\ruby[g]{身体}{からだ}を清めたり、ここへ帽子や\ruby{傘}{かさ}を預けたりするのである。
海水着を持たない私にも持物を盗まれる恐れはあったので、私は海へはいるたびにその茶屋へ\ruby[g]{一切}{いっさい}を脱ぎ\ruby{棄}{す}てる事にしていた。

\end{document}
